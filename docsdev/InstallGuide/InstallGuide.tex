

\documentclass[11pt, oneside]{article}   	% use "amsart" instead of "article" for AMSLaTeX format
\usepackage{geometry}                		% See geometry.pdf to learn the layout options. There are lots.
\geometry{letterpaper}                   		% ... or a4paper or a5paper or ... 
%\geometry{landscape}                		% Activate for for rotated page geometry
%\usepackage[parfill]{parskip}    		% Activate to begin paragraphs with an empty line rather than an indent
\usepackage{graphicx}				% Use pdf, png, jpg, or eps with pdflatex; use eps in DVI mode
\usepackage{listings}			         % For showing code			
								% TeX will automatically convert eps --> pdf in pdflatex		
\usepackage{amssymb}
\usepackage{array}
\newcolumntype{L}[1]{>{\raggedright\let\newline\\\arraybackslash\hspace{0pt}}m{#1}}
\newcolumntype{C}[1]{>{\centering\let\newline\\\arraybackslash\hspace{0pt}}m{#1}}
\newcolumntype{R}[1]{>{\raggedleft\let\newline\\\arraybackslash\hspace{0pt}}m{#1}}

\title{Installation Guide for M3SYM v1.1}
\author{Papoian Lab, University of Maryland}
\date{}							% Activate to display a given date or no date

\begin{document}
\maketitle
%\section{}
%\subsection{}

\tableofcontents
\newpage

\section{Unpacking M3SYM}
 
 To unpack the M3SYM tar file, run the following command in your terminal shell: \newline \newline \indent\texttt{> tar -xvf M3SYM.tar -C <InstallDirectory>} \newline \newline Once this is complete, all source code and other files will be in the chosen directory.
 
 
\section{Setting up the Makefile}

The Makefile for compilation of M3SYM will be in \texttt{InstallDirectory/M3SYM}, along with all source code that is needed for compilation.

\subsection {Compilers and libraries needed}

M3SYM is a C++ program that can be compiled with the following C++11 compilers:

\begin{itemize}
\item GCC 4.7 and above (Full C++11 support)
\item  Clang 3.3 and above (Also full C++11 support, default Apple compiler)
\end{itemize}  

\noindent Compiling with incomplete C++11 compatibility may result in compilation errors. \newline M3SYM uses the following math and utility libraries:

\begin{itemize}
\item Boost libraries 1.49 or above
\item  GSL library
\end{itemize} 

\subsection{Editing the Makefile}

\subsubsection{Compiler and library choices}

The Makefile can be edited to include a compiler or library in a non-default directory by changing the \texttt{CXX}, \texttt{CPPFLAGS}, and \texttt{LDLIBS} variables within the Makefile. 

\subsubsection{Optimization flags}

The code can be compiled with either \texttt{DEBUG} flags, which specifies the default debugging flags for compatibility with GDB and other debugger tools. For optimal performance, compile with the \texttt{FAST} flag, which gives a number of optimization flags. This can be edited for the system specifications.

\subsection {Command line compilation macros}

The command line macros can be edited in the Makefile to turn on or off certain code capabilities. See the Usage guide for more details on these macros and their implications. The macros available for user editing are: \newline

\small
\begin{table} [!ht]
\centering
\begin{tabular}{|L{5cm}|L{7.5cm}|}  
\hline
 \textbf{Macro} & \textbf{Description} \\
 \hline
  CHEMISTRY & Enable system chemistry \\
  \hline
  MECHANICS & Enable system mechanics \\ 
  \hline
  DYNAMICRATES & Enable dynamic rate changing. This macro can only be specified if both CHEMISTRY and
  MECHANICS are enabled. \\
  \hline
  BOOST\_MEM\_POOL & Enable boost memory pool optimizations \\
  \hline
   BOOL\_POOL\_NSIZE & Set boost memory pool size \\
  \hline
   TRACK\_DEPENDENTS & Track reaction dependents in system \\
  \hline
    TRACK\_ZERO\_COPY\_N & For activation of reactions \\
  \hline
     TRACK\_UPPER\_COPY\_N & For activation of reactions \\
  \hline
     REACTION\_SIGNALING & Enable reaction callback signaling \\
  \hline
     RSPECIES\_SIGNALING & Enable species callback signaling\\
  \hline
\end{tabular}
\end{table}
\normalsize

	
\subsection {Dependency file}

An optional dependency file can be generated by running the command  \texttt{make Makefile.dep}. This command will automatically be performed when the typical make function is executed.

\subsection{Compilation}

The code can be compiled into an executable file \texttt{M3SYM} by running \texttt{make} at the command line. \texttt{make clean} will erase all object files as well as the executable in the local directory.

\section {Running the M3SYM executable}

To run the executable, put the following command into the terminal shell: \newline \newline\indent \texttt{> ./M3SYM -s <SystemFile> -i <InputDirectory> -o <OutputDirectory>} \newline \newline More details on the system input file and directories can be found in the Usage guide.


\end{document}  