

\documentclass[11pt, oneside]{article}   	% use "amsart" instead of "article" for AMSLaTeX format
\usepackage{geometry}                		% See geometry.pdf to learn the layout options. There are lots.
\geometry{letterpaper}                   		% ... or a4paper or a5paper or ... 
%\geometry{landscape}                		% Activate for for rotated page geometry
%\usepackage[parfill]{parskip}    		% Activate to begin paragraphs with an empty line rather than an indent
\usepackage{graphicx}				% Use pdf, png, jpg, or eps with pdflatex; use eps in DVI mode
\usepackage{listings}			         % For showing code			
								% TeX will automatically convert eps --> pdf in pdflatex		
\usepackage{amssymb}

\title{Example Guide for M3SYM v1.0}
\author{Papoian Lab, University of Maryland}
\date{}							% Activate to display a given date or no date

\begin{document}
\maketitle
%\section{}
%\subsection{}

\tableofcontents
\newpage

\section{Introduction}

Examples can be found in \texttt{InstallDirectory/examples}. Each example includes a \texttt{SystemFile} as well as chemical input file. To run a given example once the M3SYM executable is created, go to the \texttt{InstallDirectory} and run the following:\newline

\indent \texttt{> ./M3SYM -s ./examples/<ExampleFolder>/system.txt}\newline
\indent\indent \texttt{-i ./examples/<ExampleFolder>/  -o <OutputDirectory>}\newline

\noindent where \texttt{<ExampleFolder>} is the specific example folder desired, and \texttt{<OutputDirectory>} is the directory of the desired output. See the usage guide for more details on these files and directories.


\section{A basic cytoskeletal network}

This example, a basic cytoskeletal network with alpha-actinin cross-linkers and myosin IIA motors, can be found in
\texttt{InstallDirectory/examples/actinnetwork}. This is set up to be a 20s simulation.

\subsection{Initial system configuration}
 
 This example is set up with the following initial configuration:
 
 \begin{itemize}
 \item Compartment size of 100nm, in a 10x10x10 grid 
 \item 1um diameter spherical boundary
 \item 100 randomly placed filaments, all initially 0.2um
 \end{itemize}

\subsection{Chemistry involved}

This example is set up with the following chemical configuration:

\begin{itemize}
\item Diffusing actin, Arp2/3, and alpha-actinin species
\item Initial concentrations of 10uM, 50nM, and 500nM respectively.
\item Actin polymerization and depolymerization
\item Alpha-actinin binding and unbinding

\end{itemize}

\noindent For more information on the reaction constants and concentrations chosen, see (*paper*).

\subsection{Mechanics involved}

This example is set up with the following mechanical configuration:

\begin{itemize}
\item Harmonic actin filament stretching force field
\item Harmonic actin filament bending force field
\item Harmonic alpha actinin cross-linker stretching force field
\item Harmonic branch stretching and cosine angle force field
\item Exponential boundary force field
\item Repulsive excluded volume force field
\end{itemize}

\noindent Dynamic reaction rate changes involving these force fields, including alpha-actinin unbinding, were tuned to fit single molecule experiments.\newline\newline
\noindent For more information on the force field parameters chosen, please see (*paper*). 


\subsection{Sample visual output}

Coming soon!

\end{document}