


\documentclass[11pt, oneside]{article}   	% use "amsart" instead of "article" for AMSLaTeX format
\usepackage{geometry}                		% See geometry.pdf to learn the layout options. There are lots.
\geometry{letterpaper}                   		% ... or a4paper or a5paper or ... 
%\geometry{landscape}                		% Activate for for rotated page geometry
%\usepackage[parfill]{parskip}    		% Activate to begin paragraphs with an empty line rather than an indent
\usepackage{graphicx}				% Use pdf, png, jpg, or eps with pdflatex; use eps in DVI mode
								% TeX will automatically convert eps --> pdf in pdflatex		
\usepackage{amssymb}
\usepackage{longtable}
\usepackage{array}
\newcolumntype{L}[1]{>{\raggedright\let\newline\\\arraybackslash\hspace{0pt}}m{#1}}
\newcolumntype{C}[1]{>{\centering\let\newline\\\arraybackslash\hspace{0pt}}m{#1}}
\newcolumntype{R}[1]{>{\raggedleft\let\newline\\\arraybackslash\hspace{0pt}}m{#1}}


\title{Usage Guide for M3SYM v1.0}
\author{Papoian Lab, University of Maryland}
\date{}							% Activate to display a given date or no date

\begin{document}
\maketitle


%\section{}
%\subsection{}

\tableofcontents

\newpage
\section {Introduction}

 Cell motility plays a key role in human biology and disease, contributing ubiquitously
 to such important processes as embryonic development, wound repair and cancer 
 metastasis. Papoian laboratory is interested in gaining deeper understanding of the
 physical chemistry behind these complex, far-from-equilibrium mechanochemical 
 processes. His approach and model, named the \textit{Mechanochemical Dynamics of Active Networks,
 3rd Generation} (MEDYAN3), is based on combining stochastic reaction-diffusion treatment
 of cellular biochemical processes with polymer physics of cytoskeletal filament network 
 growth, while explicitly coupling chemistry and mechanics. For a more detailed description of MEDYAN3, 
 please see (*paper here*)
 
 Papoian laboratory has developed M3SYM, a software package based on the MEDYAN3
 model, to simulate growth dynamics of actin based filamentous networks \textit{in vitro} and 
 \textit{in vivo}. Recent papers where M3SYM or its predecessor, StochTools, were used 
 can be found on the publication section of the Papoian group's main web page. 
 The M3SYM package can also be extended to simulate the dynamics of any active matter network.

\section {Overview of Features}

M3SYM is a package that can simultaneously simulate complex chemical and mechanical dynamics
of an active matter network. For more information on the MEDYAN3 model, which M3SYM implements,
please see (*paper*)

\subsection{Chemical capabilities}
 The chemical capabilities of M3SYM include:

\begin{itemize}

\item Stochastic reaction-diffusion on a three dimensional grid using stochastic simulation algorithms, including the \textit{Gillespie} and \textit{Next Reaction Method}.
\item Complex chemical representation of filaments, allowing for heterogeneous chemical monomers in a single filament segment.
\item A wide range of filament reactions, including:
\begin{itemize}
\item Polymerization of either end of filament
\item Depolymerization of either end of filament
\item Severing of filament at chosen sections
\item Branching of filaments
\item Cross-linker binding and unbinding to filament
\end{itemize}

\break
In the case of cytoskeletal networks, the following reactions can also be simulated:
\begin{itemize}
\item Actin filament aging by ATP hydrolysis
\item Myosin II motor binding, unbinding, and walking
\end{itemize}

\end{itemize}

\subsection{Mechanical capabilities}
M3SYM allows for a wide range of mechanical interactions, including:

\begin{itemize}
\item Force fields for filament interactions, including
\begin{itemize}
\item Filament stretching and bending
\item Branching point stretching and bending
\item Excluded volume interactions
\end{itemize}
\item Stretching force fields for cross-linkers and myosin II motors 
\item Boundary interaction force fields
\end{itemize}

\noindent These force fields can be minimized by a choice of conjugate gradient algorithm.

\subsection{Mechanochemical coupling} 

M3SYM couples chemistry and mechanics by altering reaction rates based on mechanical stresses in a given 
network. This allows for a full treatment of the complex mechanochemical responses in active matter networks.
See (*paper*) for a more detailed description.

\section{Running M3SYM}

To run the M3SYM executable, execute the following command in the terminal shell: \newline \newline\indent \texttt{> ./M3SYM -s <SystemFile> -i <InputDirectory> -o <OutputDirectory>} \newline \newline The \texttt{SystemFile} will be described in the later sections. \newline\newline  The \texttt{InputDirectory} specifies where all input files are contained, with their names being specified in the \texttt{SystemFile}. This must be an absolute directory path. The \texttt{OutputDirectory} specifies where the produced output will be placed. This also must be an absolute directory path. See the later sections for details on input and output files.
\section {Input}

\subsection{System file}

The system file is a simple text file that defines all parameters of the simulation. The M3SYM executable must take
 in a system file as a command line argument. \newline\newline Each parameter must be defined in the following syntax: \newline \newline \centerline{\texttt{<PARAMETER>:  <PARAMETERVALUE>}} \newline\newline where the parameter name is followed by a semicolon, and the value of the parameter is placed after the semicolon. Outlined below are the parameters that can be included. \\
 
 \noindent Unless otherwise noted, all distance parameters are in units of nanometers. 

\subsubsection{Geometry}

The following geometrical parameters can be set. All geometry parameters must be set in the system file, or a startup error will result. 

\begin{table} [!ht]
\centering
\begin{tabular}{|L{5cm}|C{3cm}|L{5.2cm}|}  
\hline
 \textbf{Parameter} & \textbf{Value type} & \textbf{Description} \\
 \hline
  NDIM & 1, 2, 3 & Number of dimensions in system \\
  \hline
  NX & int & Number of compartments in X \\
  \hline
  NY & int & Number of compartments in Y \\
  \hline
  NZ & int & Number of compartments in Z \\
  \hline
  COMPARTMENTSIZEX & double & Size of compartment in X \\
  \hline
  COMPARTMENTSIZEY & double & Size of compartment in Y \\
  \hline
  COMPARTMENTSIZEZ & double & Size of compartment in Z \\
  \hline
  MONOMERSIZE & double & Size of monomer for filament growth \\
  \hline
  CYLINDERSIZE & double & Size of cylinder in filament\\
  \hline
  BOUNDARYSHAPE & SPHERICAL, CUBIC, CAPSULE & Boundary shape \\
  \hline
  BDIAMETER & double & Diameter for applicable shapes, including SPHERICAL and CAPSULE \\
  \hline

\end{tabular}
\end{table}

\break
\subsubsection{Mechanics}

The following mechanical parameters can be set. It should be noted that the number of parameters for each force field must match the number of species of that type, specified in the \texttt{SystemFile}. Force field constant units are dependent on the potential used, but in general will be in pN scaling. For more information on force fields used in the MEDYAN3 model, see (*paper*). If a force field type is left blank, that force field will not be included in the simulation.

\begin{longtable}{|L{4.65cm}|C{4cm}|L{4cm}|}  
\hline
 \textbf{Parameter} & \textbf{Value type} & \textbf{Description} \\
 \hline
  CONJUGATEGRADIENT & POLAKRIBIERE, FLETCHERRIEVES, STEEPESTDESCENT & Type of conjugate gradient minimization \\
  \hline
  GRADIENTTOLERANCE & double & Gradient tolerance in conjugate gradient (in pN) \\
    \hline
  ENERGYTOLERANCE & double & Energy tolerance in conjugate gradient (in pN nm) \\
    \hline
   MAXDISTANCE & double & Maximum distance beads can be moved in minimization \\
  \hline
  FSTRETCHINGTYPE & HARMONIC & Filament stretching force field \\
  \hline
   FSTRETCHINGK & double & Filament stretching force constant \\
  \hline
    FBENDINGTYPE & HARMONIC, COSINE & Filament bending force field \\
  \hline
   FBENDINGK & double & Filament bending force constant \\
  \hline
   FBENDINGTHETA & double & Filament bending angle (radians) \\
  \hline
   LSTRETCHINGTYPE & HARMONIC & Cross-linker stretching force field \\
  \hline
   LSTRETCHINGK & double & Cross-linker stretching force constant \\
  \hline
    MSTRETCHINGTYPE & HARMONIC & Myosin II  stretching force field \\
  \hline
   MSTRETCHINGK & double & Myosin II stretching force constant \\
  \hline
   BRSTRETCHINGTYPE & HARMONIC & Branching point stretching force field \\
  \hline
   BRSTRETCHINGK & double & Branching point stretching force constant \\
  \hline
    BRBENDINGTYPE & COSINE & Branching point bending force field \\
  \hline
   BRBENDINGK & double & Branching point bending force constant \\
   \hline
   BRBENDINGTHETA & double & Branching point bending angle (radians) \\
  \hline
   BRDIHEDRALTYPE & COSINE & Branching point dihedral force field \\
  \hline
   BRDIHEDRALK & double & Branching point stretching force constant \\
  \hline
   BRPOSITIONTYPE & HARMONIC & Branching point position force field \\
  \hline
   BRPOSITIONK & double & Branching point position force constant \\
  \hline
   VOLUMETYPE & REPULSION & Volume force type \\
  \hline
   VOLUMECUTOFF & double & Volume interaction cutoff distance\\
  \hline
  VOLUMEK & double & Volume force constant \\
  \hline
  BOUNDARYTYPE & REPULSIONEXP, REPULSIONLJ & Boundary force type \\
  \hline
   BOUNDARYCUTOFF & double & Boundary interaction cutoff distance\\
  \hline
  BINTERACTIONK & double & Boundary force constant \\
  \hline
  BSCREENLENGTH & double & Boundary screening length constant\\
  \hline
 
\end{longtable}

\subsubsection{Chemistry}

The following chemical parameters can be set. The number of species of each type must match the chemistry input file, as well as the number of mechanical parameters for each force field. All chemical parameters must be set unless otherwise noted in the description. For the motor parameters, the number of parameters must match the number of motor species in the system. For more information on chemical algorithms, see (*paper*)

\begin{longtable}{|L{4.9cm}|C{2.5cm}|L{4.75cm}|}  
\hline
 \textbf{Parameter} & \textbf{Value type} & \textbf{Description} \\
 \hline
  CHEMISTRYFILE & string & Input chemistry file. Should be in the \texttt{InputDirectory} \\
  \hline
  CALGORITHM & GILLESPIE, NRM & Chemistry algorithm used \\
  \hline
  NUMTOTALSTEPS & int & Number of total chemical steps to perform. If RUNTIME is set, this will not be used \\
  \hline
  RUNTIME & double & Total runtime of simulation (seconds) \\
  \hline
  NUMSTEPSPERS & int & Number of total steps per snapshot. If SNAPSHOTTIME is set, this will not be used \\
  \hline
  SNAPSHOTTIME & double & Time of each snapshot (seconds) \\
  \hline
  NUMCHEMSTEPS & int & Number of chemical steps per mechanical equilibration \\
  \hline
  NUMSTEPSPERN & int & Number of chemical steps per neighbor list update. This includes updating
   chemical reactions as well as force fields which rely on neighbor lists. \\
  \hline
  NUMDIFFUSINGSPECIES & int & Diffusing species in system \\
  \hline
  NUMBULKSPECIES & int & Bulk species in system \\
  \hline
  NUMFILAMENTSPECIES & int & Filament species in system \\
  \hline
  NUMPLUSENDSPECIES & int & Plus end species in system \\
  \hline
  NUMMINUSENDSPECIES & int & Minus end species in system \\
  \hline
  NUMBOUNDSPECIES & int & Bound species in system \\
  \hline
    NUMLINKERSPECIES & int & Cross-linker species in system \\
  \hline
    NUMMOTORSPECIES & int & Myosin II motor species in system \\
  \hline
    NUMBRANCHERSPECIES & int & Brancher species in system \\
  \hline
    NUMBINDINGSITES & int & Number of binding sites per cylinder. This will set binding sites for cross-linkers,
    motors, and other binding molecules. \\
  \hline
    NUMMOTORHEADSMIN & int & Minimum number of motor heads per motor species defined. \\
    \hline
    NUMMOTORHEADSMAX & int & Maximum number of motor heads per motor species defined. \\
    \hline
    MOTORSTEPSIZE & double & Single motor head step size. \\
    \hline
 
\end{longtable}

\subsubsection{Dynamic rates}

The following dynamic rate forms and parameters can be set. These parameters are characteristic lengths and amplitudes of the rate changing equations outlined in (*paper*). These can be tuned to mimic the stall and unbinding mechanochemical coupling of cross-linkers and myosin II motors. Note that if dynamic rates are enabled, the number of dynamic rate forms for each type of reaction must match the number of species of that type specified in the \texttt{SystemFile}, i.e. the number of forms for cross-linker unbinding must match the number of cross-linker species, etc.

The number of parameters specified for each type of dynamic rate form must match the number of parameters required for those forms. See below for details, and see (*paper*) for more information on the explicit forms. Parameters must be listed in order of the form that they correspond to, also corresponding to the species that they represent.

\begin{longtable}{|L{5.5cm}|C{3.2cm}|L{6cm}|}  
\hline
 \textbf{Parameter} & \textbf{Value type} & \textbf{Description} \\
 \hline
 DFPOLYMERIZATIONTYPE & BROWRATCHET & Filament polymerization dynamic rate form \\
 \hline
  DFPOLYMERIZATIONLEN & double & Characteristic length for filament polymerization dynamic rate form \\
  \hline
  DLUNBINDINGTYPE & BASICCATCHSLIP, BASICSLIP & Cross-linker unbinding dynamic rate form. 
  If BASICCATCHSLIP, two
  parameters for DLUNBINDINGLEN and DLUNBINDINGAMP are needed to define the functional form. 
  If BASICSLIP, one 
  DLUNBIDINGLEN is needed to define the functional form. \\
  \hline
  DLUNBINDINGLEN & double & Characteristic length of cross-linker unbinding dynamic rate form \\
  \hline
   DLUNBINDINGAMP & double & Amplitude of cross-linker unbinding dynamic rate form \\
  \hline
    DMUNBINDINGTYPE & LOWDUTYPCMCATCH & Myosin II unbinding dynamic rate form. \\
  \hline
  DMUNBINDINGFORCE & double & Characteristic force of myosin II unbinding dynamic rate form \\
  \hline
  DMWALKINGTYPE & LOWDUTYHILLSTALL & Myosin II walking dynamic rate form \\ 
  \hline
  DMWALKINGLEN & double & Characteristic force of myosin II walking dynamic rate form \\
  \hline
 
\end{longtable}

\subsubsection{Starting filament configuration}

The following filament initialization parameters can be set. These parameters define the initial configuration and length
of filaments in the system. It is noted that at least one filament, plus end, and minus end chemical species must be initialized in the chemistry input file, or a startup error will result.

\begin{table} [!ht]
\centering
\begin{tabular}{|L{4cm}|C{4cm}|L{6cm}|}  
\hline
 \textbf{Parameter} & \textbf{Value type} & \textbf{Description} \\
 \hline
  FILAMENTFILE & string & Name of filament initialization file. This is not required.\\
  \hline
  NUMFILAMENTS & int & Number of random filaments to initialize. These filaments will be randomly 
  distributed in the system volume.\\
  \hline
  FILAMENTLENGTH & int & Number of cylinders per filament to initialize, defining the initial length of the filaments.\\
  \hline

\end{tabular}
\end{table}

\subsubsection{Output formats}

The output of M3SYM will be directed to the \texttt{OutputFile} specified. The following output can be set. These outputs must be set on different lines, so users should specify a new parameter, on separate lines, for each output value desired. Output files will be explained in more detail in a later section.

\begin{table} [!ht]
\centering
\begin{tabular}{|L{4cm}|C{6cm}|L{4cm}|}  
\hline
 \textbf{Parameter} & \textbf{Value type} & \textbf{Description} \\
 \hline
  OUTPUTTYPE & SNAPSHOT, FORCES, STRESSES, BIRTHTIMES & Output type\\
  \hline
 
\end{tabular}
\end{table}

\subsection{Chemistry input file}
The chemistry input file, whose name is specified in the \texttt{SystemFile}, contains the chemical configuration of the system, including species and reactions. It is noted that the order in which cross-linker, motor, and branches species are defined in the chemistry input file should match the relevant mechanical parameters, which are defined in the \texttt{SystemFile}. The number of species of each type should also match the \texttt{SystemFile}'s species type numbers, or a startup error will result.

\subsubsection{Species}
Different types of species can be defined as follows:

\begin{itemize}
\item 
A \textbf{diffusing species} is defined in the following form:\newline\newline \centerline{\texttt{SPECIESDIFFUSING: <NAME> <COPYNUMBER> <DIFFRATE> <RELEASETIME> <QUALIFIER> <NUMEVENTS>}}\newline\newline where \texttt{NAME} is any string defining the name of the species, \texttt{COPYNUMBER} is the number of molecules of that species in the system, and \texttt{DIFFUSIONRATE} is a float value that determines the diffusion rate of this molecule between compartments. \texttt{RELEASETIME} specifies when this molecule populates the system in simulation (in seconds).\\

The \texttt{QUALIFIER} field is used to define the type of reacting species. The options are the following:
\begin{itemize}
\item \texttt{REG} : A regular reacting species. Copy numbers are updated typically.
\item \texttt{AVG} : An averaging reacting species. The species will use a copy number averaged over a set number of copy number changes (\texttt{NUMEVENTS}) for efficiency.
\end{itemize}
The \texttt{NUMEVENTS} field is only used in the case of defining an averaging reacting species. If using a regular, this should  not be included in the file or an error will result.

\item
A \textbf{bulk species}, which is assumed to be spatially homogeneous, is defined in the following form:\newline\newline\centerline{\texttt{SPECIESBULK: <NAME> <COPYNUMBER> <RELEASETIME> <QUALIFIER>}}\newline\newline where \texttt{NAME} is any string defining the name of the species, \texttt{COPYNUMBER} is the number of molecules of that species in the system, and \texttt{RELEASETIME} specifies when this molecule populates the system in simulation (in seconds).

The \texttt{QUALIFIER} field is used to define the type of reacting species. The options are the following:
\begin{itemize}
\item \texttt{REG} : A regular reacting species. Copy numbers are updated typically.
\item \texttt{CONST} : An constant reacting species. The species will never change copy number upon reacting.
\end{itemize}

\item
Any \textbf{filament-related species} can be defined in the following form:\newline\newline\centerline{\texttt{SPECIES<SPECIESTYPE>: <NAME>}}\newline\newline where \texttt{SPECIESTYPE} can be:
\begin{itemize}
\item \texttt{FILAMENT} : A filamentous species. At least one filament species must be defined if using filaments in simulation.
\item \texttt{PLUSEND} : A plus end species on a filament, which is defined as the front of the filament. There must be at least one plus end species for every filament species defined in the system.
\item \texttt{MINUSEND} : A minus end species on a filament, which is defined as the back of the filament. There must be at least one minus end species for every filament species defined in the system.
\item \texttt{BOUND} : A bound species on a filament. There must be at least one bound species defined in the system, which will denote the "empty" spot on a filament.
\item \texttt{LINKER} : A cross-linker species. The ordering of cross-linker initializations should match their mechanical parameters, as stated above.
\item \texttt{MOTOR} : A myosin II motor species. The ordering of motor initializations should match their mechanical parameters, as stated above.
\item \texttt{BRANCHER} : A branching species. The ordering of branches initializations should match their mechanical parameters, as stated above.
\end{itemize}
\end{itemize}

\subsubsection{Reactions}

Reaction definitions must follow these common rules:
\begin{itemize}
\item Species that are defined in reactions must be previously defined in the chemistry file. 
\item For filament-related reactions, most species type and ordering parameters are fixed; if they are fixed, they will be pre-defined in the reaction definition below. If the ordering is not properly followed, a startup error will result. 
\item All species declarations in a reaction must be separated by white space, with \texttt{+} markers between reactants and products. A \texttt{->} must be placed between reactants and products, separated by whitespace. If this syntax is not followed, a startup error will result.
\end{itemize}

\noindent Different types of reactions can be defined as follows:
\begin{itemize}


\item A \textbf{general reaction} between any bulk or diffusing species can be defined in the following form: \newline\texttt{GENREACTION:}\newline 
\texttt{<NAME>:BULK/DIFFUSING + <NAME>:BULK/DIFFUSING + ... ->}\newline
\texttt{<NAME>:BULK/DIFFUSING + <NAME>:BULK/DIFFUSING + ... <RATE>}\newline\newline 
where any bulk or diffusing species can be included, and \texttt{<RATE>} is a float value that determines the rate constant of the reaction.

\item A \textbf{bulk reaction} between bulk species only can be defined in the following form: \newline\newline
\texttt{BULKREACTION:}\newline
\texttt{<NAME>:BULK + <NAME>:BULK + ... ->}\newline
\texttt{ <NAME>:BULK + <NAME>:BULK + ... <RATE>}\newline\newline 
where any bulk species can be included. If the reaction only contains bulk species, it must be specified as a bulk reaction. \texttt{<RATE>} is a float value that determines the rate constant  of the reaction.

\item A \textbf{polymerization reaction} can be defined in the following form:\newline\newline
\texttt{POLYMERIZATIONREACTION:}\newline
\texttt{<NAME>:BULK/DIFFUSING + <NAME>:PLUSEND/MINUSEND ->}\newline
\texttt{<NAME>:FILAMENT + <NAME>:PLUSEND/MINUSEND <RATE>}\newline\newline 
where \texttt{<NAME>} is the string name of the species, and \texttt{<RATE>} is a float value that determines the rate constant  of the reaction. It is noted that the first species listed can be either \texttt{DIFFUSING} or \texttt{BULK}, and the reaction can contain a \texttt{PLUSEND} or \texttt{MINUSEND}.

This reaction will polymerize the filament, producing a new chemical species on the end of the filament and increasing 
the length of the filament by a single monomer.

\item A \textbf{depolymerization reaction} can be defined in the following form:\newline\newline
\texttt{DEPOLYMERIZATIONREACTION:}\newline
\texttt{<NAME>:FILAMENT + <NAME>:PLUSEND/MINUSEND ->}\newline
\texttt{<NAME>:BULK/DIFFUSING + <NAME>:PLUSEND/MINUSEND <RATE>}\newline\newline
 where \texttt{<NAME>} is the string name of the species, and \texttt{<RATE>} is a float value that determines the rate constant of the reaction. It is noted that the third species listed can be either \texttt{DIFFUSING} or \texttt{BULK}, and the reaction can contain a \texttt{PLUSEND} or \texttt{MINUSEND}.
 
 This reaction will depolymerize the filament, removing a chemical species from the end of the filament and decreasing the length of the filament by a single monomer.

\item A \textbf{cross-linker reaction} between two filaments can be defined in the following form:\newline\newline
\texttt{LINKERREACTION:}\newline
\texttt{<NAME>:BOUND:1 + <NAME>:BOUND:2 + <NAME>:BULK/DIFFUSING <->}\newline
\texttt{<NAME>:LINKER:1 + <NAME>:LINKER:2 <ONRATE> <OFFRATE> <RMIN> <RMAX>}\newline

where \texttt{<NAME>} is the string name of the species, and \texttt{<ONRATE>} and \texttt{<OFFRATE>} are float values that determines the rate constant of the binding and unbinding reactions. \texttt{<RMIN>} and \texttt{<RMAX>} are the range of the chemical reaction, and this can be set depending on the structure of the simulated cross-linker. It is noted that the third species listed can be either \texttt{DIFFUSING} or \texttt{BULK}. 

This reaction produces cross-linker species at two separate positions on each respective filament which are chemically and mechanically connected. If mechanical force fields are defined for the cross-linkers, a potential will be created between the filaments. The unbinding reaction will remove these species from the filaments, as well as remove any linker potentials that have been created between the filaments.

 
 \item A \textbf{myosin II motor reaction} between two filaments can be defined in the following form:\newline\newline
\texttt{MOTORREACTION:}\newline
\texttt{<NAME>:BOUND:1 + <NAME>:BOUND:2 + <NAME>:BULK/DIFFUSING <->}\newline
\texttt{<NAME>:MOTOR:1 + <NAME>:MOTOR:2 <ONRATE> <OFFRATE> <RMIN> <RMAX>}\newline

where \texttt{<NAME>} is the string name of the species, and \texttt{<ONRATE>} and \texttt{<OFFRATE>} are float values that determines the rate constant of the binding and unbinding reactions. \texttt{<RMIN>} and \texttt{<RMAX>} are the range of the chemical reaction, and this can be set depending on the structure of the simulated motor. It is noted that the third species listed can be either \texttt{DIFFUSING} or \texttt{BULK}. 

This binding reaction produces motor species at two separate positions on each respective filament which are chemically and mechanically connected. If mechanical force fields are defined for the motor, a potential will be created between the filaments. The unbinding reaction will remove these species from the filaments, as well as remove any motor potentials that have been created between the filaments.


\item A \textbf{myosin II motor walking reaction} can be defined in the following form:\newline\newline
\texttt{MOTORWALKINGREACTION:}\newline
\texttt{<NAME>:MOTOR:N/N+1 + <NAME>:BOUND:N/N+1 ->}\newline
\texttt{<NAME>:MOTOR:N/N+1 + <NAME>:BOUND:N/N+1 <RATE>}\newline\newline
where \texttt{<NAME>} is the string name of the species, and \texttt{<RATE>} is a float value that determines the rate constant of the reaction. The choice of \texttt{N/N+1} will determine whether the motor is stepping forward or backward. A motor movement from \texttt{N} to \texttt{N+1} is defined as forward movement (towards the plus end of the filament), and the opposite is backward (towards the minus end). These choices for the reactants and products must be self-consistent as well as consistent with the bound species positions chosen in the reaction, or a startup error will result. 

This reaction will move a motor head in the given direction.
 
\item A \textbf{branching reaction} can be defined in the following form:\newline\newline
\texttt{BRANCHINGREACTION:}\newline
\texttt{<NAME>:BULK/DIFFUSING + <NAME>:BULK/DIFFUSING +  <NAME>:BOUND <->}\newline
\texttt{<NAME>:BRANCHER + <NAME>:PLUSEND <RATE>}\newline\newline
where \texttt{<NAME>} is the string name of the species, and \texttt{<RATE>} is a float value that determines the rate constant of the reaction.  It is noted that the first and second species listed can be either \texttt{DIFFUSING} or \texttt{BULK}.

This reaction will create a new branching point, as well as a filament with the desired chemical plus end. If mechanical force fields are defined for the branching point, a potential will be created between the parent and child filament. The unbinding reaction will remove the branching point from the filaments, thus freeing the child filament from the parent. It will also remove any branching point potentials that have been created between the filaments.
 
\item A \textbf{nucleation reaction} can be defined in the following form:\newline\newline
\texttt{NUCLEATIONREACTION:}\newline
\texttt{<NAME>:BULK/DIFFUSING + <NAME>:BULK/DIFFUSING ->}\newline
\texttt{<NAME>:PLUSEND + <NAME>:FILAMENT + <NAME>:MINUSEND <RATE>}\newline\newline
where \texttt{<NAME>} is the string name of the species, and \texttt{<RATE>} is a float value that determines the rate constant of the reaction. It is noted that the first and second species listed can be either \texttt{DIFFUSING} or \texttt{BULK}.

This reaction will create a new filament with the given chemical plus end, minus end, and filament species.

\item A \textbf{destruction reaction} can be defined in the following form:\newline\newline
\texttt{DESTRUCTIONREACTION:}\newline
\texttt{<NAME>:PLUSEND + <NAME>:MINUSEND ->}\newline
\texttt{<NAME>:BULK/DIFFUSING + <NAME>:BULK/DIFFUSING <RATE>}\newline\newline
where \texttt{<NAME>} is the string name of the species, and \texttt{<RATE>} is a float value that determines the rate constant of the reaction. It is noted that the third and fourth species listed can be either \texttt{DIFFUSING} or \texttt{BULK}.

This reaction will destroy a filament, removing it from the system.

\item An \textbf{filament aging reaction} can be defined in the following form:\newline\newline
\texttt{AGINGREACTION:}\newline
\texttt{<NAME>:FILAMENT/PLUSEND/MINUSEND ->}\newline
\texttt{<NAME>:FILAMENT/PLUSEND/MINUSEND <RATE>}\newline\newline
where \texttt{<NAME>} is the string name of the species, and \texttt{<RATE>} is a float value that determines the rate constant of the reaction. Either of the reactant or product species can be \texttt{FILAMENT}, \texttt{PLUSEND}, or \texttt{MINUEND}, but the product and reactant species must be the same type, or a startup error will result.

This reaction will change the chemical species that resides in a filament.

\item A \textbf{filament severing reaction} can be defined in the following form:\newline\newline
\texttt{SEVERINGREACTION:}\newline
\texttt{AT <NAME>:FILAMENT <RATE>}\newline\newline
where \texttt{<NAME>} is the string name of the species, and \texttt{<RATE>} is a float value that determines the rate constant of the reaction.

This reaction will sever the filament at the closest cylinder connection to a given chemical position, producing two child filaments.

\end{itemize}

\subsection{Filament input file}

The filament input file, whose name is specified in the \texttt{SystemFile}, contains coordinates of filaments to initialize in the system at startup. The format of an initial filament declaration is as follows:\newline

\texttt{FILAMENT: coord1x coord1y coord1z coord2x coord2y coord2z}\newline\newline
\noindent where \texttt{\{coord1x, coord1y, coord1z\}} and \texttt{\{coord2x, coord2y, coord2z\}} specify the starting and 
ending coordinates of the filament.

\section {Output}

M3SYM can produce a number of output types, set in the \texttt{SystemFile}, produced at a snapshot frequency also defined in the \texttt{SystemFile}. These output files will be placed in the \texttt{OutputDirectory} specified at runtime. The output types and visualization of this output are described below.

\subsection{Types of output files}
\subsubsection{snapshot.traj}
The snapshot file gives the basic trajectory information of the system. It includes a brief description for all filaments, cross-linkers, motors, and branching points in the system, as well as information on the current chemical step. It is produced with the following form:\newline

\noindent\texttt{chemstepnumber time numfilaments numlinkers nummotors numbranchers}\newline
\texttt{F filamentid filamentcyllength deltal deltar}\newline
\texttt{beadcoord1x beadcoord1y beadcoord1z beadcoord2x beadcoord2y beadcoord2z ...}\newline
\texttt{...}\newline
\texttt{L linkerid linkertype }\newline
\texttt{startcoordx startcoordy startcoordz endcoordx endcoordy endcoordz}\newline
\texttt{...}\newline
\texttt{M motorid motortype }\newline
\texttt{startcoordx startcoordy startcoordz endcoordx endcoordy endcoordz}\newline
\texttt{...}\newline
\texttt{B brancherid branchertype }\newline
\texttt{coordx coordy coordz}\newline

\subsubsection{forces.traj, stresses.traj, and birthtimes.traj}

The forces file gives the forces on each element in the system, in similar form to the snapshot file. It is produced with the following format:\newline

\noindent\texttt{chemstepnumber time numfilaments numlinkers nummotors numbranchers}\newline
\texttt{F filamentid filamentcyllength deltal deltar}\newline
\texttt{bead1property bead2property ...}\newline
\texttt{...}\newline
\texttt{L linkerid linkertype }\newline
\texttt{linkerproperty}\newline
\texttt{...}\newline
\texttt{M motorid motortype }\newline
\texttt{motorproperty}\newline
\texttt{...}\newline
\texttt{B brancherid branchertype }\newline
\texttt{*no property printed for branching points*}\newline
\texttt{...}\newline

\noindent where the properties are as follows:
\begin{itemize}
\item forces.traj: the magnitude forces on each cylinder, as well as the magnitude of stretching force on each cross-linker and motor are printed.
\item stresses.traj: the stretching stress on cylinders, cross-linkers, and motors are printed.
\item birthtimes.traj: the birth time of on cylinders, cross-linkers, and motors are printed.

\end{itemize}

\subsection{Visualization of output}

The output described in the previous section can be visualized using a python script found in \texttt{InstallDirectory/visual}, named \texttt{ReadTrajectory.py}. This script uses MayaVi (http://mayavi.sourceforge.net) to produce a visualization of trajectory frames, as well as an animation of an entire simulation.

\subsubsection{Installation of MayaVi}

The following python-related dependencies for MayaVi should be installed:
\begin{itemize}
\item vtk5 5.10.1
\item qt4 4.8.6
\item ipython 2.2.0
\item matplotlib 1.4.0
\item pyside 1.2.2
\end{itemize}

\noindent Most of these packages are available through MacPorts or Homebrew, if using an Apple computer. The following environment variables
may need to be declared, depending on your system configuration:\newline

\texttt{export QT\_API = pyside (or qt4)}\newline

\noindent A helpful alias to run ipython is:\newline

\texttt{alias im = "ipython --gui = qt --pylab = qt"}\newline

\noindent We will assume for the next section that this alias is configured.


\subsubsection{Running the visualization script, ReadTrajectory.py}

The visualization script must be edited by the user on lines 6-8 to include the desired snapshot file. A color file can also be specified, which will color the network based on either a force, stress, or birth time file. If no color file is included, the elements of the network will be colored to a default value. The script can also be edited on lines 201-214 to include titles, scales, default colors, and choice of color map. \newline

\noindent Run the script using the following commands:\newline\newline
\indent\texttt{> ip}\newline
\indent\texttt{> run -i ReadTrajectory}\newline

\noindent This will load all snapshots of the trajectory files specified. To show a frame, execute the following commands:\newline

\texttt{> show\_frame(frame number)}\newline

\noindent To show the entire simulation frame by frame, execute the following:\newline

\texttt{> anim()}

\section{Developer guide}

\subsection{Pull requests}

Pull requests for the M3SYM repository should be directed to James Komianos (jkomianos@gmail.com). From there, detailed instructions on cloning the repository, and general guidelines on adding features will be given.

\subsection{Coding style}

M3SYM follows standard style guides and object-oriented design principles. For details on general C++ coding style, see the Google C++ style guide at http://google-styleguide.googlecode.com/svn/trunk/cppguide.html. In general, the following best practices should be followed:

\begin{itemize}

\item Use camelCase for all declarations, with classes beginning with a capital letter
\item Use proper tabs for all declarations, loops, control flow, etc.
\item Keep code under 80 characters per line

\end{itemize}

\noindent For documentation, please use Doxygen-style commenting. See the Doxygen user guide at 
http://www.stack.nl/~dimitri/doxygen/ for more information.

\subsection{Other information}

Please direct all code inquiries to James Komianos (jkomianos@gmail.com).

\end{document}  