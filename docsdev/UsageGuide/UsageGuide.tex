


\documentclass[11pt, oneside]{article}   	% use "amsart" instead of "article" for AMSLaTeX format
\usepackage{geometry}                		% See geometry.pdf to learn the layout options. There are lots.
\geometry{letterpaper}                   		% ... or a4paper or a5paper or ... 
%\geometry{landscape}                		% Activate for for rotated page geometry
%\usepackage[parfill]{parskip}    		% Activate to begin paragraphs with an empty line rather than an indent
\usepackage{graphicx}				% Use pdf, png, jpg, or eps with pdflatex; use eps in DVI mode
								% TeX will automatically convert eps --> pdf in pdflatex		
\usepackage{amssymb}
\usepackage{array}
\newcolumntype{L}[1]{>{\raggedright\let\newline\\\arraybackslash\hspace{0pt}}m{#1}}
\newcolumntype{C}[1]{>{\centering\let\newline\\\arraybackslash\hspace{0pt}}m{#1}}
\newcolumntype{R}[1]{>{\raggedleft\let\newline\\\arraybackslash\hspace{0pt}}m{#1}}


\title{Usage Guide for M3SYM v1.0}
\author{Papoian Lab, University of Maryland}
\date{}							% Activate to display a given date or no date

\begin{document}
\maketitle


%\section{}
%\subsection{}

\tableofcontents

\newpage
\section {Introduction}

 Cell motility plays a key role in human biology and disease, contributing ubiquitously
 to such important processes as embryonic development, wound repair and cancer 
 metastasis. Papoian laboratory is interested in gaining deeper understanding of the
 physical chemistry behind these complex, far-from-equilibrium mechano-chemical 
 processes. His approach and model, named \textit{Mechano-chemical Dynamics of Active Networks,
 3rd Generation} (MEDYAN3), is based on combining stochastic reaction-diffusion treatment
 of cellular biochemical processes with polymer physics of cytoskeletal filament network 
 growth, while explicitly coupling chemistry and mechanics. For a more detailed description of MEDYAN3, 
 please see (*paper here*)
 
 Papoian laboratory has developed M3SYM, a software package based on the MEDYAN3
 model, to simulate growth dynamics of actin based filamentous networks \textit{in vitro} and 
 \textit{in vivo}. Recent papers where M3SYM or its predecessor, StochTools, were used 
 can be found on the publication section of the Papoian group's main web page. 
 The M3SYM package can also be extended to simulate the dynamics of any active matter network.

\section {Overview of Features}

M3SYM is a package that can simultaneously simulate complex chemical and mechanical dynamics
of an active matter network. For more information on the MEDYAN3 model, which M3SYM implements,
please see (*paper*)

\subsection{Chemical capabilities}
 The chemical capabilities of M3SYM include:

\begin{itemize}

\item Stochastic reaction-diffusion on a three dimensional grid using stochastic simulation algorithms, including the \textit{Gillespie} and \textit{Next Reaction Method}.
\item Complex chemical representation of filaments, allowing for heterogeneous chemical monomers in a single filament segment.
\item A wide range of polymer reactions, including:
\begin{itemize}
\item Polymerization of either end of filament
\item Depolymerization of either end of filament
\item Severing of filament at chosen sections
\item Branching of filament
\item Cross-linker binding and unbinding to filament
\end{itemize}

\break
In the case of cytoskeletal networks, the following reactions can also be simulated:
\begin{itemize}
\item Actin filament aging by ATP hydrolysis
\item Myosin IIA binding, unbinding, and walking
\end{itemize}

\end{itemize}

\subsection{Mechanical capabilities}
M3SYM allows for a wide range of mechanical interactions, including:

\begin{itemize}
\item Force fields for filament interactions, including
\begin{itemize}
\item Filament stretching and bending
\item Branching point stretching and bending
\item Excluded volume interactions
\end{itemize}
\item Stretching force fields for cross-linkers and motors 
\item Boundary interaction force fields
\end{itemize}

\noindent These force fields can be minimized by a choice of conjugate gradient algorithm.

\subsection{Mechano-chemical coupling} 

M3SYM couples chemistry and mechanics by altering reaction rates based on mechanical stresses in a given 
network. This allows for a full treatment of the complex mechano-chemical responses in active matter networks.
See (*paper*) for a more detailed description.

\section{Running M3SYM}

To run the M3SYM executable, put the following command into the terminal shell: \newline \newline \centerline{\texttt{./M3SYM -s <SystemFile> -i <InputDirectory> -o <OutputDirectory>}} \newline \newline The \texttt{SystemFile} will be described in the later sections. \newline\newline  The \texttt{InputDirectory} specifies where all input files are contained. This must be an absolute directory path. The \texttt{OutputDirectory} specifies where the produced output will be placed. This also must be an absolute directory path. See the later sections for details on input and output files.
\section {Input}

\subsection{System file}

The system file is a simple text file that defines all parameters of the simulation. The M3SYM executable must take
 in a system file as a command line argument. \newline\newline Each parameter must be defined in the following syntax: \newline \newline \centerline{\texttt{[PARAMETER]:  [PARAMETERVALUE]}} \newline\newline where the parameter name is followed by a semicolon, and the value of the parameter is placed after the semicolon. Outlined below are the parameters that can be included. 

\subsubsection{Geometry}

The following geometry parameters can be set:

\begin{table} [!ht]
\centering
\begin{tabular}{|L{5cm}|C{3cm}|L{6.8cm}|}  
\hline
 \textbf{Parameter} & \textbf{Value type} & \textbf{Description} \\
 \hline
  NDIM & 1, 2, 3 & Number of dimensions in system \\
  \hline
  NX & int & Number of compartments in X \\
  \hline
  NY & int & Number of compartments in Y \\
  \hline
  NZ & int & Number of compartments in Z \\
  \hline
  COMPARTMENTSIZEX & double & Size of compartment in X \\
  \hline
  COMPARTMENTSIZEY & double & Size of compartment in Y \\
  \hline
  COMPARTMENTSIZEZ & double & Size of compartment in Z \\
  \hline
  MONOMERSIZE & double & Size of monomer for filament growth \\
  \hline
  CYLINDERSIZE & double & Size of cylinder in filament\\
  \hline
  BOUNDARYSHAPE & SPHERICAL, CUBIC, CAPSULE & Boundary shape \\
  \hline
  BDIAMETER & double & Diameter for applicable shapes, including SPHERICAL and CAPSULE \\
  \hline

\end{tabular}
\end{table}

\break
\subsubsection{Mechanics}

The following mechanics parameters can be set. It should be noted that the number of parameters for each force field must match the number of chemical species of that type. Force field constant units are dependent on the potential used. For more information on force fields used, see (*paper*)
\newline\newline
\noindent If a force field type is left blank, that force field will not be included in the simulation.

\begin{table} [!ht]
\centering
\begin{tabular}{|L{5cm}|C{4cm}|L{7cm}|}  
\hline
 \textbf{Parameter} & \textbf{Value type} & \textbf{Description} \\
 \hline
  CONJUGATEGRADIENT & POLAKRIBIERE, FLETCHERRIEVES & Type of conjugate gradient minimization \\
  \hline
  GRADIENTTOLERANCE & double & Gradient tolerance in conjugate gradient \\
  \hline
  FSTRETCHINGTYPE & HARMONIC & Filament stretching force field \\
  \hline
   FSTRETCHINGK & double & Filament stretching force constant \\
  \hline
    FBENDINGTYPE & HARMONIC, COSINE & Filament bending force field \\
  \hline
   FBENDINGK & double & Filament bending force constant \\
  \hline
   FBENDINGTHETA & double & Filament bending angle \\
  \hline
   LSTRETCHINGTYPE & HARMONIC & Cross-linker stretching force field \\
  \hline
   LSTRETCHINGK & double & Cross-linker stretching force constant \\
  \hline
    MSTRETCHINGTYPE & HARMONIC & Motor stretching force field \\
  \hline
   MSTRETCHINGK & double & Motor stretching force constant \\
  \hline
   BRSTRETCHINGTYPE & HARMONIC & Branching point stretching force field \\
  \hline
   BRSTRETCHINGK & double & Branching point stretching force constant \\
  \hline
    BRBENDINGTYPE & COSINE & Branching point bending force field \\
  \hline
   BRBENDINGK & double & Branching point bending force constant \\
   \hline
   BRBENDINGTHETA & double & Branching point bending angle \\
  \hline
   BRDIHEDRALTYPE & COSINE & Branching point dihedral force field \\
  \hline
   BRDIHEDRALK & double & Branching point stretching force constant \\
  \hline
   BRPOSITIONTYPE & HARMONIC & Branching point position force field \\
  \hline
   BRPOSITIONK & double & Branching point position force constant \\
  \hline
   VOLUMETYPE & REPULSION & Volume force type \\
  \hline
   VOLUMECUTOFF & double & Volume interaction cutoff distance\\
  \hline
  VOLUMEK & double & Volume force constant \\
  \hline
  BOUNDARYTYPE & REPULSIONEXP, REPULSIONLJ & Boundary force type \\
  \hline
   BOUNDARYCUTOFF & double & Boundary interaction cutoff distance\\
  \hline
  BINTERACTIONK & double & Boundary force constant \\
  \hline
  BSCREENLENGTH & double & Boundary screening length constant\\
  \hline
 
 \end{tabular}
\end{table}

\break
\subsubsection{Chemistry}

The following chemistry parameters can be set. It should be noted that the number of chemical species of each type must match the chemistry input file, as well as the number of mechanical parameters for each force field (if defined).

\begin{table} [!ht]
\centering
\begin{tabular}{|L{5cm}|C{4cm}|L{6.8cm}|}  
\hline
 \textbf{Parameter} & \textbf{Value type} & \textbf{Description} \\
 \hline
  CHEMISTRYFILE & string & Input chemistry file. Should be in the \texttt{InputDirectory} \\
  \hline
  CALGORITHM & SIMPLEGILLESPIE, GILLESPIE, NRM & Chemistry algorithm used \\
  \hline
  NUMTOTALSTEPS & int & Number of total chemical steps to perform. If RUNTIME is set, this will not be used \\
  \hline
  RUNTIME & double & Total runtime of simulation (seconds) \\
  \hline
  NUMSTEPSPERS & int & Number of total steps per snapshot. If SNAPSHOTTIME is set, this will not be used \\
  \hline
  SNAPSHOTTIME & double & Time of each snapshot (seconds) \\
  \hline
  NUMCHEMSTEPS & int & Number of chemical steps per mechanical equilibration \\
  \hline
  NUMSTEPSPERN & int & Number of chemical steps per neighbor list update. This includes updating
   chemical reactions as well as force fields which rely on neighbor lists. \\
  \hline
  NUMDIFFUSINGSPECIES & int & Number of diffusing species in system \\
  \hline
  NUMBULKSPECIES & int & Number of bulk species in system \\
  \hline
  NUMFILAMENTSPECIES & int & Number of filament species in system \\
  \hline
  NUMPLUSENDSPECIES & int & Number of plus end species in system \\
  \hline
  NUMMINUSENDSPECIES & int & Number of minus end species in system \\
  \hline
  NUMBOUNDSPECIES & int & Number of bound species in system \\
  \hline
    NUMLINKERSPECIES & int & Number of linker species in system \\
  \hline
    NUMMOTORSPECIES & int & Number of motor species in system \\
  \hline
    NUMBRANCHERSPECIES & int & Number of brancher species in system \\
  \hline
    NUMBINDINGSITES & int & Number of binding sites per cylinder. This will set binding sites for cross-linkers,
    motors, and other binding molecules. \\
  \hline
 
\end{tabular}
\end{table}

\subsubsection{Dynamic rates}

The following dynamic rate parameters can be set. These parameters are characteristic lengths of the rate changing equations outlined in (*paper*). These can be tuned to mimic the stall and unbinding forces of cross-linkers and motors.

\begin{table} [!ht]
\centering
\begin{tabular}{|L{5cm}|C{4cm}|L{6cm}|}  
\hline
 \textbf{Parameter} & \textbf{Value type} & \textbf{Description} \\
 \hline
  FDPLENGTH & double & Characteristic length for filament polymerization rate change \\
  \hline
  MDULENGTH & double & Characteristic length for motor unbinding rate change \\
  \hline
  MDWLENGTH & double & Characteristic length for motor walking rate change \\
  \hline
 LDULENGTH & double & Characteristic length for cross-linker unbinding rate change \\
  \hline
 
\end{tabular}
\end{table}

\subsubsection{Output formats}

The output of M3SYM will be directed to the \texttt{OutputFile} specified. The following output can be set. These outputs must be set on different lines, so users should specify a new OUTPUTTYPE parameter for each value desired. Output files will be explained in more detail in a later section.

\begin{table} [!ht]
\centering
\begin{tabular}{|L{5cm}|C{3cm}|L{6cm}|}  
\hline
 \textbf{Parameter} & \textbf{Value type} & \textbf{Description} \\
 \hline
  OUTPUTTYPE & SNAPSHOT, FORCES, STRESSES, BIRTH TIMES & Output type\\
  \hline
 
\end{tabular}
\end{table}


\subsection{Chemistry input file}
\subsubsection{Species}
\subsubsection{Reactions}

\subsection{Filament input file}



\section {Output}
\subsection{Types of output files}
\subsection{Visualization of output}

\section{For Developers}
\subsection{Style guide}
\subsection{Unit testing}


\section{Contact}


\end{document}  